\subsection{Dell 2}

Paper \cite{dell-2} 


% ----------------------------------------------------------------
\subsubsection{Abstract}

\begin{itemize}
    \item Homomorphism polynomials enumerate all homomorphisms from a pattern graph $H$ to $n$-vertex graphs
    \item looking for a pattern graph H in another graph G, called the host graph -> homomorphism
    \item looking for subgraphs of host graph G which are isomorphic to H
    \item host + pattern as input -> NP-complete
    \item pattern graph fixed size -> better possible
    \item suffices to consider the hom polys from H to $K_n$ as the homomorphism polynomial of $H$
    \item arithmetic circuit constructions of hom polys can be used to obtain almost all known better algorithms for detecting induced subgraph isomorphisms as well
    \item hom poly yield poly families that are complete for classes VP (bounded treewidth) and VNP (?)
    \item Graph parameter: Treewidth, parthwidth, treedepth (from being a star)
    \item H has bounded treewidth, then there are small-sized ari circ for hom polys
    \item ari circ constructions of hom polys based on treewidth do not use negative constants -> monotone
    \item Schnorr: expo lower bound for clique poly (special case hom poly: H is clique)
\end{itemize}

% ----------------------------------------------------------------
\subsubsection{contributions}

\begin{itemize}
    \item treew, pathw and treed exactly characterize the complexity of hom poly for arithmetic circuits, ABPs and arithmetic formulas
    \item colored subgraph iso polys instead of hom polys
    \item Q: in how many monomial computations can a single gate participate? A: dictated by treewidth
    \item comp polynomial and use it to construct tree decomposition
    \item gate can participate in the computation of at most $n^{k-tw(H)-1}$ monomials (degree $k$)
    \item monotone circ can not produce invalid submonomial (no cancellations)
    \item treewidth of a clique on $n$ vertices is $n-1$
\end{itemize}


% Theorems
\begin{theorem}
    The monotone arithmetic circuit complexity of homomorphism polynomial \todo{plural?}for a pattern graph $H$ is $\Theta(n^{tw(H)+1})$, where $tw(H)$ is the treewidth of $H$.
\end{theorem}

\begin{theorem}
    The monotone ABP complexity of homomorphism polynomial for a pattern graph $H$ is $\Theta(n^{pw(H)+1})$, where $pw(H)$ is the pathwidth of $H$.
\end{theorem}

\begin{theorem}
    The monotone formula complexity of homomorphism polynomial for a pattern graph $H$ is $\Theta(n^{pw(H)+1})$, where $pw(H)$ is the treedepth of $H$.
\end{theorem}

% ----------------------------------------------------------------
\subsubsection{Preliminaries}

\begin{itemize}
    \item for any poly: size of smallest ABP and size of smallest skew circuit are within constant factors of each other.
    \item colored iso poly: enumerates all col iso from patttern to host where there are $n$ vertics of each color. This polynomial can be used to count col iso in $n$-vertex host graphs by setting the variables corresponding to edges not in the host graph to $0$.
    \item size of a tree decomposition is the size of the largest bag minus one.
    \item Treeewidth is the size of a smallest tree decomposition of H.
    \item if path: path decomposition (pathwidth)
    \item For all graphs $tw \leq pw \leq td - 1$
    \item for $p\geq 2$ there is a tree $X_k$ on $k=2^{p+1}-1$ vertices that have pathwidth p.
    \item all paths have pathwidth 1 and the k-vertex path has treedepth $\lceil \log_2(k+1)\rceil$.
\end{itemize}


% Definitions
\begin{definition}\label{def:mono}
    A polynomial over $\mQ$ is called monotone if all its coefficients are non-negative.
\end{definition}

\begin{definition}\label{def:ari-circ}
    An arithmetic circuit over the variables $x_1,\dots,x_n$ is a rooted DAG \todo{directed, acyclic Graph ausschreiben?} where each source node (also called an input gate) is labeled by one of the variables $x_i$ or a constant $a\in\mQ$. All other nodes (called gates) are labeled with either $+$ (addition) or $\times$ (multiplication). The circuit computes a polynomial over $\mQ[x_1,\dots,x_n]$ in the usual fashion.\\
    The circuit is called monotone if all constnats are non-negative.\\
    The circuit is a skew circuit if for all $\times$ gates, at least one of the inputs is a variable or a constant.\todo{what else??}\\
    The circuit is a formula if all gates have out-degree at most one.\\
    The size of a circuit or skew circuit or formula is the number of edges in the circuit.\\
    The depth of a circuit is the number of gates in the longest path from the root to an input gate.
\end{definition}

\begin{definition}
    \todo{ABP notwendig?}
    An Algebraic Branching Program (ABP) is a DAG with a unique source ndoe $s$ and a unique sink node $t$. Each edge is labeled with a variable from $x_1,\dots,x_n$ or a constant $a\in\mQ$. Each path in the DAG from $s$ to $t$ corresponds to a term obtained by multiplying all edge labels on that path. The polynomial computed by the ABP is the sum of all terms over all paths from $s$ to $t$. The ABP is called monotone if all constants are non-negative. The size of the ABP is the number of edges.
\end{definition}

\begin{definition}\label{def:hom}
    For graphs $H$ and $G$, a homomorphism from $H$ to $G$ is a function $\phi:V(H)\mapsto V(G)$ such that $\{i,j\}\in E(H)$ implies $\{\phi(i),\phi(j)\}\in E(G)$. For an edge $e=\{i,j\}$ in $H$, we use $\phi(e)$ to denote $\{\phi(i),\phi(j)\}$.
\end{definition}

\begin{definition}\label{def:col-iso}
    Let $H$ be a $k$-Vertex graph where its vertices are labeled $[k]$ and let $G$ be a graph where each vertex has a color in $[k]$. Then, a colored isomorphism of $H$ in $G$ is a subgraph of $G$ isomorphic to $H$ such that all vertices in the subgraph have different colors and for each edge $\{i,j\}$ in $H$, there is an edge in the subgraph between vertices colored $i$ and $j$.
\end{definition}

\begin{definition}\label{def:hom-poly}
    For a pattern graph $H$ on $k$ vertices, the $n^{\text{th}}$ homomorphism polynomial for $H$ is a polynomial on $\binom{n}{2}$ variables $x_e$ where $e=\{u,v\}$ for $u,v\in [n]$.
    \[\text{Hom}_{H,n}=\sum\limits_{\phi}\prod\limits_e x_{\phi(e)}\]
    where $\phi$ ranges over all homomorphisms from $H$ to $K_n$ and $e$ ranges over all edges in $H$.
\end{definition}

\begin{definition}\label{def:col-iso-poly}
    For a pattern graph $H$ on $k$ vertices, the $n^{\text{th}}$ \textit{colored isomorphism polynomial} for $H$ is a polynomial on $|E(H)|n^2$ variables $x_e$ where $e=\{(i,u),(j,v)\}$ for $u,v\in [n]$ and $\{i,j\}\in E(H)$.
    \[ \text{Collso}_{H,n}=\sum\limits_{u_1,\dots,u_k}\prod\limits_{i,j} x_{\{(i,u_i),(j,u_j)\}} \]
    where $u_1,\dots,u_k\in [n]$ and $\{i,j\}\in E(H)$.
\end{definition}

\noindent
New labeling can be obtained by the substitution $x_{\{(i,u),(j,v)\}}\mapsto x_{\{(\xi(i),u),(\xi(j),v)\}}$.

\begin{definition}
    \todo{Notwendig?}
    Let $g$ be a gate in a circuit $C$. A \textit{parse tree} rooted at $g$ is any rotted tree which can be obtained by the following procedure, duplicating gates in $C$ as neccessary to preserve the tree structure.
    \begin{enumerate}
        \item The gate $g$ is the root of the tree.
        \item If there is a multiplication gate $g$ in the tree, include all its children in the circuit as its children in the tree.
        \item If there is an addition gate $g$ in the tree, pick an arbitrary child of $g$ in the circuit and include it in the tree.
    \end{enumerate}
\end{definition}

\begin{definition}\label{def:tree-decomposition}
    A \textit{tree decomposition} of $H$ is a tree where each vertex (called a \textit{bag}) in the tree is a subset of vertices of $H$. This tree must satisy two properties.
    \begin{enumerate}
        \item For every edge $\{i,j\}$ in $H$, there must be at least one bag in the tree that contains both $i$ and $j$.
        \item for any vertex $i$ in $H$, the subgraph of the tree decomposition induced by all bags containing $i$ must be a subtree. This subtree is called the \textit{subtree induced by $i$}.
    \end{enumerate}
\end{definition}

\begin{definition}
    \todo{Notwendig?}
    For a connected graph $H$, an \textit{elimination tree} of $H$ is a rooted, directed tree that can be constructed by arbitrarily picking a vertex $u$ in $H$ and adding edges from the roots of elimination trees of connected components of $H-u$ to the root vertex labeled $u$. In particular, if $H$ is a single vertex, then the elimination tree of $H$ is the same single vertex graph.\\
    The \textit{depth} of an elimination tree is the number of vertices in the longest path from a leaf to the root. The \textit{treedepth} of $H$ is the depth of the smallest depth elimination tree of $H$.
\end{definition}

% ----------------------------------------------------------------
\subsubsection{Algebraic complexity of homomorphism polynomials}

\begin{itemize}
    \item Hom image of $P_3$ will either be another $P_3$ or an edge $P_2$.\\
    
\end{itemize}

\begin{definition}
    A \textit{separating set} for a polynomial $p$ is a set of monomials in $p$ such that for any two monomials $s$ and $t$ in the seperating set, there does not exist a monomial $m$ of $p$ such that $m$ divides the product $st$.
\end{definition}

\noindent 
The size of a separating set lower bounds the size of the monotone arithmetic circuit computing the polynomial. $\text{Hom}_{K_k}$ requires monotone arithmetic circuits of size $n^k$.

\begin{theorem}
    \todo{Proposition}
    Any separating set for $\text{Hom}_{P_3,n}$ has size at most $n$ for sufficiently large $n$.
\end{theorem}

Proof...

\noindent
Hom image of $P_3$ will either be another $P_3$ or an edge $P_2$.\\
monomials of Hom$_{P_3,n}$ correspond to $P_3$ or edge.

\begin{theorem}
    \todo{Proposition}
    Any separating set for Hom$_{C_4,n}$ has size at most $n^2$ for sufficiently large $n$.
\end{theorem}

Proof...

\begin{theorem}\label{thm:mon-circ-complexity}
    The \textit{monotone circuit complexity} of Hom$_{H}$ is $\Theta(n^{\text{tw}+1})$.
\end{theorem}

Proof... Über tree-decomposition \todo{verstehen}